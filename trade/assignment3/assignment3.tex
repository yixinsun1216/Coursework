\documentclass{article}
\usepackage[utf8]{inputenc}
\usepackage[english]{babel}
\usepackage{amsmath} 
\usepackage{amsfonts}
\usepackage{pdfpages}
\usepackage{bbm}
\usepackage{pdfpages}
\usepackage{graphicx}
\usepackage[margin=1in]{geometry}
\usepackage{hyperref}
\usepackage{dsfont}
\usepackage{natbib}
\usepackage{tabularx}
\usepackage[flushleft]{threeparttable}
\usepackage{booktabs}
\usepackage{pdflscape}
\usepackage{graphicx}
%\usepackage{subfig}
\usepackage{subcaption}

\hypersetup{
    colorlinks=true,
    linkcolor=blue,
	filecolor=magenta,   
	citecolor = black,   
    urlcolor=blue,
}


%\setlength\parindent{0pt}

\title{Assignment 3}
\author{Yixin Sun}

\begin{document}
\maketitle

\section*{Urban Theory}

In Henderson (1974) there are a continuum of equilibria. In Behrens, Duranton, Robert-Nicoud (2014), Proposition 4 says there is a unique talent-homogeneous equilibrium. Is Proposition 4 correct? If it is correct, very clearly explain why talent heterogeneity makes the equilibrium unique when the homogeneous-worker case in Henderson (1974) yields multiple equilibria. If it is incorrect, identify the error and state the correct claim.

Proposition 4 from Behrens, Duranton, Robert-Nicoud (2014): If $\gamma / \varepsilon$ is close to unity, the talent-homogeneous equilibrium is unique and such that 
$$
L(t)=\left(\frac{1+\gamma}{1+\varepsilon} \xi t^{1+a}\right)^{1 /(\gamma-\varepsilon)}
$$
where
$$
\xi \equiv \frac{(\varepsilon \sigma)^{1+\varepsilon} S^{1+a}}{\gamma \theta}
$$

Following their proof, each individual solves a constrained optimization problem that consists of picking the city with talent $t$ that maximizes her expected indirect utility from the menu of possible cities. They solve for the equation that determines the menu of talents and populations that supports a talent-homogenous equilibrium:
\begin{align}
    \gamma \theta L(t)^{\varepsilon}\left[\frac{\xi t^{1+a}-L(t)^{\gamma-\varepsilon}}{L(t)} d L(t)+\frac{1+a}{1+\varepsilon} \xi t^{a} d t\right]=0
\end{align}


\section*{Trade and Urban Theory}
Look at Behrens and Robert-Nicoud's "Agglomeration Theory with Heterogeneous Agents" chapter in the Handbook of Urban and Regional Economics. On page 204, they propose a theory of metropolitan specialization in which different cities are home to different industries. What is their prediction? How can this be investigated empirically? What is the role of comparative advantage? Is comparative advantage sufficient for the existence of a specialized equilibrium?

Proposition 4.2 (industrial specialization):  Assume that $\rho=1, \nu=0,$ and $\mathbb{A}_{i c}=\mathbb{A}_{i}$ for all $i$ and all $c .$ Then all cities are specialized by industry at the unique spatial equilibrium with competitive land developers, and their size is optimal:
$$
L_{i}=\left(p_{i} \frac{\epsilon}{\gamma} \mathbb{A}_{i}\right)^{\frac{1}{\gamma-c}}
$$


In this setting, 
\end{document}


