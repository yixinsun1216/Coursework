\documentclass{article}
\usepackage[utf8]{inputenc}
\usepackage[english]{babel}
\usepackage{amsmath} 
\usepackage{pdfpages}
\usepackage{bbm}
\usepackage{pdfpages}
\usepackage{graphicx}
\usepackage[margin=1in]{geometry}
\usepackage{hyperref}
\usepackage{dsfont}
\usepackage{natbib}
\usepackage{tabularx}
\usepackage[flushleft]{threeparttable}
\usepackage{booktabs}
\usepackage{pdflscape}
\usepackage{graphicx}
%\usepackage{subfig}
\usepackage{subcaption}

\hypersetup{
    colorlinks=true,
    linkcolor=blue,
	filecolor=magenta,   
	citecolor = black,   
    urlcolor=blue,
}


%\setlength\parindent{0pt}

\title{Comprehension Check 3}
\author{Yixin Sun}

\begin{document}
\maketitle

\section*{Question 1}
In the paper, the authors define population of a city by
\begin{gather*}
    L(c)=S(\bar{\tau}(c)) = \pi\bar{\tau}^2\\
    \Rightarrow \bar{\tau} = \sqrt{\frac{L}{\pi}}
\end{gather*}
The paper also shows, from Lemma 1, that 
\begin{align*}
    \underline{\gamma} &\equiv A(c) T(\bar{\tau}(c)) \\
    &= 1 \times (d_1 - d_2\bar{\tau}) \\
    &= d_1 - d_2\sqrt{\frac{L}{\pi}}
\end{align*}

\section*{Question 2}
From Lemma 3, we are given the function for rental rates:
\begin{align*}
    r_{\Gamma}(\gamma)=  \int_{\underline{\gamma}}^{\gamma} G(K(x)) d x
\end{align*}
Combining this with Lemma 2 and the given functional form for $G(\omega)$, we can rewrite this as 
\begin{align*}
    r_{\Gamma}(\gamma) = \int_{\underline{\gamma}}^{\gamma}  gF^{-1}\left(\frac{L-S_{\Gamma}(x)}{L}\right) dx
\end{align*}
In this single country case, we can rewrite $S_\Gamma(x)$ 
\begin{align*}
    S_\Gamma(\gamma) &= S\left(T^{-1}\left(\frac{\gamma}{A(c)}\right)\right) \\
    &= S(T^{-1}(\gamma)) \\
    &= S\left(\frac{d_1 - \gamma}{d_2}\right) \\
    &= \pi\left(\frac{d_1 - \gamma}{d_2}\right)^2
\end{align*}

We want to evaluate this for the $\gamma$ such that $\tau = 0$. Call this $\bar{\gamma}$. Plugging in given equations, we get 
\begin{align*}
    \bar{\gamma} &= d_1 - d_2\underline{\tau} \\
    &= d_1
\end{align*}
Thus the rental rate can be written as
\begin{align*}
    r(\bar{\gamma})=\int_{\underline{\gamma}}^{d_1} g F^{-1}\left(\frac{L-\pi\left(\frac{d_1 - \gamma}{d_2}\right)^2}{L}\right) d x
\end{align*}
We can now use the fact that 
\begin{align*}
    \omega \sim Unif[\underline{\omega}, \bar{\omega}] &\Rightarrow F(\omega)\equiv \frac{\omega-\underline{\omega}}{\bar{\omega}-\underline{\omega}} \\
    &\Rightarrow F^{-1}(y) = y(\bar{\omega}-\underline{\omega})+\underline{\omega}
\end{align*}

We finally get the expression
\begin{align*}
    r(\bar{\gamma}) &= \int_{\underline{\gamma}}^{d_1} g\left[\left(\frac{L-\pi\left(\frac{d_{1}-x}{d_{2}}\right)^{2}}{L}\right)(\bar{\omega}-\underline{\omega})+\underline{\omega}\right] d x \\
    &= \int_{\underline{\gamma}}^{d_1} g (\bar{\omega}-\underline{\omega}) dx + \int_{\underline{\gamma}}^{d_1} \underline{\omega} dx - \frac{g\pi}{d_2^2L}\int_{\underline{\gamma}}^{d_1} (d_1 - x)^2dx
\end{align*}

\section*{Question 3}
First note that we know that $r(\underline{\gamma}) = 0$. Using this and our results from part 1, we can solve for $\underline{\omega}$. 

\begin{align*}
    r(\underline{\gamma}) = 0 &= g\left[\left(\frac{L-\pi\left(\frac{d_{1}-\underline{\gamma}}{d_{2}}\right)^{2}}{L}\right)(\bar{\omega}-\underline{\omega})+\underline{\omega}\right] \\
    &= g\left[\left(\frac{L-\pi\left(\frac{d_{1}-\left(d_1 - d_2\sqrt{\frac{L}{\pi}}\right)}{d_{2}}\right)^{2}}{L}\right)(\bar{\omega}-\underline{\omega})+\underline{\omega}\right]  \\
    \Rightarrow &= \underline{\omega} = 0
\end{align*}

If $g$ increases, the whole equilibrium rent schedule shifts upwards.  the rent at the edge of the city doesn't change since $\underline{\omega} = 0$ and thus $g\underline{\omega} = 0$, same as before. Therefore, the rent schedule also rotates counterclockwise for this to hold.

The equilibrium utility for skill level $\omega$ is:
\begin{align*}
    U(c, \tau, \sigma ; \omega)&=A(c) T(\tau) H(\omega, \sigma) p(\sigma)-r(c, \tau) 
\end{align*}
We have defined 
\begin{align*}
    G(\omega) \equiv H(\omega, M(\omega)) p(M(\omega))
\end{align*}
So we can rewrite utility as 
\begin{align*}
    U(c, \tau, \sigma ; \omega)&=T(\tau) G(\omega)-r(c, \tau) \\
    &=\left(d_{1}-d_{2} \tau\right) g \omega-\int_{\underline{\gamma}}^{\gamma} g\left[\left(\frac{L-\pi\left(\frac{d_{1}-x}{d_{2}}\right)^{2}}{L}\right) \bar{\omega}\right] d x
\end{align*}
Thus increasing $g$ should lead to a proportional increase in the utility. 

\section*{Question 4}
From the expression in part 3, we can write the utility of $\bar{\omega}$ as
\begin{align*}
    U(c, \tau, \sigma ; \bar{\omega})=\left(d_{1}-d_{2} \tau\right) g \bar{\omega}-\int_{\underline{\gamma}}^{\gamma} g\left[\left(\frac{L-\pi\left(\frac{d_{1}-x}{d_{2}}\right)^{2}}{L}\right)(\bar{\omega}-\underline{\omega})+\underline{\omega}\right] d x
\end{align*}
So we see that an increase in $\underline{\omega}$ would lead to a $g\left[\left(\frac{L-\pi\left(\frac{d_{1}-\gamma}{d_{2}}\right)^{2}}{L}\right)\right]$ decrease for each $\gamma$. We know $\pi, L > 0$ as given by the problem, ... \textcolor{blue}{Intuition is utility should decrease but how the heck do I sign this??}

\section*{Question 5}
\begin{align*}
    U(c, \tau, \sigma ; \underline{\omega})&=\left(d_{1}-d_{2} \tau\right) g \underline{\omega}-g\left[\left(\frac{L-\pi\left(\frac{d_{1}-\underline{\gamma}}{d_{2}}\right)^{2}}{L}\right)(\bar{\omega}-\underline{\omega})+\underline{\omega}\right] \\
    &= \left(d_{1}-d_{2} \tau\right) g \underline{\omega} - g\underline{\omega}
\end{align*}
So $\bar{\omega}$ does not factor into $\underline{\omega}$'s utility function. 

\textcolor{blue}{Is what I am doing ok here???}
\end{document}


